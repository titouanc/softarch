\section{Libraries}

In order to parse JSON files more easily we use the google-gson library (\href{https://github.com/google/gson}{https://github.com/google/gson}). It allows us to convert Java objects directly to JSON and vice-versa.

\section{Implementation of the JSON database layer}

Main problem : data classes are intrisically depend on SQL softarch.portal.data.* constructs SQL queries (ex: Book) -\char62{} our implementation in json uses reflection to store data -\char62{} the sql implementation should do the same -\char62{} abstract the notion of a database in an interface Database -\char62{} abstract the operations on Raw/Regular/User databases in interfaces -\char62{} define the way to store data in JSON/SQL abstract classes -\char62{} define Raw/Regular/User JSON/SQL databases that inherits and implements those above -\char62{} choose the method of storage using dsn in servlet configuration and a database factory -\char62{} to define a new method of storage : create an abstract class that define the way to store data, create three concrete class that inherit it and implements Raw/Regular/User and modify add method to the database factory to parse the new dsn.

We had to change to modify the facades constructor to take the dsn into account during the database construction. That is the only modification of other layers we did.

\section{Other design flaws}

\subsection{Data layer}

\subsubsection{XML}
Several data types have a method to create an XML representation of themselves to be used in the UI layer. The right way to do it would be to construct the XML in the UI layer. This let the UI layer alone handle the data format to send to the client and it is thus easier to change the said data format. This could be accomplished easily using the accessors of the data types.

\subsubsection{Servlet Request}
Several data types have a constructor from a `HttpServletRequest`. This binds the data to the way information is transmitted between the client and the server. The right way to do it would be for the UI layer to extract the data from the request before use in the data types.

\subsubsection{ResultSet}
Several data types have a constructor from a `ResultSet`. The data is then coupled with the way it is stored in the database. The right way to solve that problem is for the database layer to extract the data from the result set before use in the data types.

\subsubsection{URI}
Several data types have method(s) that return the HTTP URI of their location on the server. The UI layer should manage URIs and bind them to the correct resources.

% softarch.portal.data.* constructs XML (ex: Book) -\char62{} the UI layer should do it

% softarch.portal.data.* is instantiable using a Servlet Request (ex: Book) -\char62{} the UI layer should deconstruct the request and instantiate with parameters directly

% softarch.portal.data.* is instantiable using a SQL ResultSet (ex: Book) -\char62{} the database layer shoud deconstruct result sets and instantiate with parameters

% softarch.portal.data.* returns http routes (ex: Administrator) -\char62{} the UI layer should construct the routes

\subsection{Application layer}

There is no design flaw in the application layer. It manages only the session and defer every data request to the data layer.

\subsection{UI layer}

Several pages (`AdministrationPage` for example) handle queries whereas they should only extract data from the HTTP request and send it to the application layer.

`RegistrationPage` handles the creation of different types of user whereas it should only extract the useful data and pass it to the application layer (which should then let the database layer handle it).

`InternetFrontPage` handles a DatabaseException whereas the UI layer should not know anything about the data layer and let the application layer handle it.


% AdministrationPage handles query but the Application Layer should do it 
% InternetFrontPage handles a DatabaseException but the Application Layer should do it 
% Same with other pages that handle query 
% RegistrationPage handles the creation of different types of user but should defer it to the application layer (which could use a factory pattern for exp) 
% QueryPage, OperationPage, LogoutPage, LoginPage ok -\char62{} defers to application layer
