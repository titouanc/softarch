\section{Libraries}

In order to parse JSON files more easily we use the google-gson library (\url{https://github.com/google/gson}). It allows us to convert Java objects directly to JSON and vice-versa.

\section{Implementation of the JSON database layer}

The main problem with the original implementation (Figure \ref{fig:before}) is a tight coupling between the data types, the database layer and the UI layer. For example data types can be constructed from sql cursors or HTTP requests. We didn't attempt to change the whole architecture, but we added the \texttt{matchQuery(String)} and \texttt{isOlderThan(Date)} methods to the concrete regular data types. These methods allow us to perform filtering on objects loaded from json files (as we don't have an SQL-like query language).

\begin{figure}[p]
  \includegraphics[width=\textwidth]{before.eps}
  \caption{\label{fig:before}The original database layer}
\end{figure}

\subsection{Extracting interfaces}
The first architectural change in the database layer was to move the original databases classes in a new package \texttt{softarch.portal.db.sql}. Then we created interfaces with the same methods as in the original classes in the \texttt{softarch.portal.db} package. Of course, the SQL implementations now implements these interfaces. Then we created their json equivalents in the \texttt{softarch.portal.db.json} package, as represented in Figure \ref{fig:after}.

\begin{figure}[p]
  \includegraphics[width=\textwidth]{after.eps}
  \caption{\label{fig:after}The new database layer}
\end{figure}

\subsection{Database Factory}
The next main change was to implement a factory to create databases using the
right implementation. We therefore added a class \texttt{DatabaseFactory}, which
has 3 methods to create the 3 type of specialized databases. The factory choose
the actual implementation using a DSN\footnote{Data Source Name} String. This string
loooks like an URL, and for now supports the following schemes:
\begin{itemize}
  \item \texttt{hsql://username:password@db/hsql\_uri}
  \item \texttt{json:///path/to/json/directory}
\end{itemize}

Using DSN strings, we can encode different options type \textit{(path for json, database credentials for hsql, ...)}, it is easily human-readable, and can be read and written to a file without effort. We replaced the \texttt{db\_url, db\_user, db\_pass} configurations parameters with \texttt{db\_dsn} in the deployment configuration file (\texttt{WebContent/WEB-INF/web.xml}). The only thing one has to do to add a new database layer is to create a new implementation of each of the specialized databases and modify the \texttt{DatabaseFactory} to handle the new DSN. This configuration value could be changed when needed. We had to propagate the modification to the database credentials to all the classes that need to be created in order to obtain a concrete database, but from now on no more change would be required, thanks to the database facade that encapsulate the creation and access to the databases.


\section{Other design flaws}

\subsection{Data layer}

\subsubsection{XML}
Several data types have a method to create an XML representation of themselves to be used in the UI layer. The right way to do it would be to construct the XML in the UI layer. This let the UI layer alone handle the data format to send to the client and it is thus easier to change the said data format. This could be accomplished easily using the accessors of the data types.

\subsubsection{Servlet Request}
Several data types have a constructor from a \texttt{HttpServletRequest}. This binds the data to the way information is transmitted between the client and the server. The right way to do it would be for the UI layer to extract the data from the request before use in the data types.

\subsubsection{ResultSet}
Several data types have a constructor from a \texttt{ResultSet}. The data is then coupled with the way it is stored in the database. The right way to solve that problem is for the database layer to extract the data from the result set before use in the data types.

\subsubsection{URI}
Several data types have method(s) that return the HTTP URI of their location on the server. The UI layer should manage URIs and bind them to the correct resources.

% softarch.portal.data.* constructs XML (ex: Book) -\char62{} the UI layer should do it

% softarch.portal.data.* is instantiable using a Servlet Request (ex: Book) -\char62{} the UI layer should deconstruct the request and instantiate with parameters directly

% softarch.portal.data.* is instantiable using a SQL ResultSet (ex: Book) -\char62{} the database layer shoud deconstruct result sets and instantiate with parameters

% softarch.portal.data.* returns http routes (ex: Administrator) -\char62{} the UI layer should construct the routes

\subsection{Application layer}

There is no design flaw in the application layer. It manages only the session and defer every data request to the data layer.

\subsection{UI layer}

Several pages (\texttt{AdministrationPage} for example) handle queries whereas they should only extract data from the HTTP request and send it to the application layer.

\texttt{RegistrationPage} handles the creation of different types of user whereas it should only extract the useful data and pass it to the application layer (which should then let the data layer handle it).

\texttt{InternetFrontPage} handles a \texttt{DatabaseException} whereas the UI layer should not know anything about the data layer and let the application layer handle it.


% AdministrationPage handles query but the Application Layer should do it 
% InternetFrontPage handles a DatabaseException but the Application Layer should do it 
% Same with other pages that handle query 
% RegistrationPage handles the creation of different types of user but should defer it to the application layer (which could use a factory pattern for exp) 
% QueryPage, OperationPage, LogoutPage, LoginPage ok -\char62{} defers to application layer
