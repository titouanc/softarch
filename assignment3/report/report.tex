\section{Exercise 1 : BPEL Processes}

\subsection{Correct the error about the PortType}
\subsubsection{What is the Port Type of a Webservice ? Why is it important for the LibrarySearch BPEL process ?}
A web service Port Type is the abstract interface for a web service component. It define the name, input, failure and output of the operations a webservice of this port type should implement. These implementations are provided through bindings.

\subsubsection{Fix the error. Give a brief summary of your changes in the report. Document changes in other files, if necessary}

We had to change the port type of the \texttt{InvokeSearchForBooks} task in the BPEL process to the one defined in the National Library description (\url{http://localhost:8181/NationalLibrary/services/LibraryService?wsdl}): \texttt{LibraryService}. Since it is imported in the \texttt{nl} namespace in our BPEL file, we have to write \texttt{nl:LibraryService}.

\subsubsection{Using BPEL constructs, explain what happens in AssignResultSoftLib near the end of the process file}
In this task, we assign the result value to the response object, just before sending it to the caller. The result value is built from the responses of the two other services, described in a transformation language: XSLT.

\subsection{Notice that the SOFT Library only returns books in French, even though it has English books as well}
\subsubsection{Find out why this is the case, and correct the problem so that SOFT Library searcher returns books in English as well. Document the solution in the report}

The BPEL process uses an xsl transformation stylesheet to combine the ouptuts from the two providers libraries into one. One of the template matching tags is an XPath query for all books in english (\texttt{<xsl:template match="//book[language='en']"></xsl:template>}). As this tag is empty, it replaces any book in english with nothing, and therefore we only see books in french. However, if the Soft Lab Library contained books in other languages than french or english, we would see them in the query result.

To obtain books in all language, it is sufficient to remove this tag, as the previous rewrite rule will match and return all nodes.

\subsubsection{Analyse the data that the two web services return. Are they the same? Why could this be a problem in a system? What is a possible solution ?}

The data from the two library webservices are nearly identical but differ in two ways:
\begin{itemize}
  \item The SOFT library return a \texttt{searchBooksResponse} containing a list of \texttt{book}, while the National Library returns a list of \texttt{searchForBooksReturn}, each of which contains exactly 1 \texttt{book}.
  \item The SOFT library gives the publication year of the books, but the National Library give the full date of publication.
\end{itemize}

This is a problem because our application has to process distinctly the results of the two sources, even if the intermediate BPEL process aggregates them. We propose 2 solutions to this problem:

\begin{itemize}
  \item The SOFT and national libraries both aggree on a common web service port type, and therefore return their records in the same format. This is however not likely to happen because two different products have to be modified and the complexity of this process increases as more libraries would want to participate.
  \item The XSL transformation occuring at the end of the BPEL process could be enhanced to normalize the data from the two source in a coherent book list.
\end{itemize}


\section{Exercise 2 : Integration with Legacy Software}
\subsection{What did you change in the web portal application?}
We choosed to add the webservice access in the database layer. We added a new package: \texttt{softarch.portal.db.webservice}, which only contains a \texttt{WebServiceRegularDatabase}, analogous to the \texttt{JSONRegularDatabase} from previous assignment. This class only implements two methods: \texttt{findRecords} and \texttt{getNumberOfRegularRecords}, and raise \texttt{DatabaseException} for all others operations. Finally, the \texttt{DatabaseFacade} class holds a private reference to a \texttt{WebServiceDatabase}, and add results from this service when its implemented methods are used on the database.

\subsection{How did your refactoring from the last assignment help or hinder you with this task ?}
We already had an interface for a \texttt{RegularDatabase}. We only had to provide a new implementation, and plug it directly to the \texttt{DatabaseFacade}. However, we did not add a factory method for this new class, because the database facade always has to create a concrete object of this class.

\subsection{Which pattern or architectural style did you use to make this task (if any) ?}
We used the service-oriented achitecture, as we were asked to combine the result of other fixed webservices into our new one.

\subsection{With respect to second part of Exercise 1 (Analysing the results of the two web services): How did you integrate the results in the web portal application ?
Which element in your architecture is responsible, and what are the benefits and drawbacks of your design ?}
Our new Webservice \texttt{RegularDatabase} implementation calls the Books Web Service, then transform its result in a list of \texttt{softarch.data.Book}, as required by the interface. It is therefore the only component affected by any change in the Books web service.

\section{Exercise 3 : Architecture}

    • Chose an appropriate visualization, for example a UML or block diagram. The
    diagram should allow to see the connections between the relevant elements in this
    architecture, e.g., the involved web services, and the relevant parts of the web
    portal application. Also minimally depict your architecture which results from
    your design choices in exercise 2.

    • Briefly describe your diagram and name the relevant concepts/architectural styles.
    Do not go into the details for responsible classes etc., the architecture has to show
    logical components, not primarily implementation details. Often you will have no
    control over available web services, so treat the library web services as if they where
    provided by external providers.
